\documentclass[UTF8]{article}
\usepackage{ctex}
\usepackage{geometry}
\geometry{a4paper, scale=0.8}
\usepackage{graphicx}
\usepackage{subfigure}
\usepackage{float}
\usepackage{bookmark}
\usepackage{fontspec}
\newfontfamily\jetbrains{JetBrains Mono}
\usepackage{listings}
\usepackage{color}
\lstset{
    breaklines,                                 % 自动将长的代码行换行排版
    extendedchars=false,                        % 解决代码跨页时,章节标题,页眉等汉字不显示的问题
    backgroundcolor=\color[rgb]{0.96,0.96,0.96},% 背景颜色
    keywordstyle=\color{blue}\bfseries,         % 关键字颜色
    identifierstyle=\color{black},              % 普通标识符颜色
    commentstyle=\color[rgb]{0,0.6,0},          % 注释颜色
    stringstyle=\color[rgb]{0.58,0,0.82},       % 字符串颜色
    showstringspaces=false,                     % 不显示字符串内的空格
    numbers=left,                               % 显示行号
    numberstyle=\tiny\jetbrains,                    % 设置数字字体
    basicstyle=\small\jetbrains,                    % 设置基本字体
    captionpos=t,                               % title在上方(在bottom即为b)
    frame=single,                               % 设置代码框形式
    rulecolor=\color[rgb]{0.8,0.8,0.8},         % 设置代码框颜色
}
\usepackage{enumitem}

\title{Verilog实验\ lab4实验报告}
\author{PB19071405\ 王昊元}
\date{2022 年 05 月 16 日}

\begin{document}
    \maketitle
    \section{实验目的}
    \begin{enumerate}
        \item 实现BTB(Branch Target Buffer)和BHT(Branch History Table)两种动态分支预测器
        \item 体会动态分支预测对流水线性能的影响
    \end{enumerate}
    \section{实验要求}
    \begin{enumerate}
        \item 在Lab3阶段二的RV32I Core基础上,实现BTB
        \item 在阶段一的基础上实现BHT
    \end{enumerate}
    \section{实验环境}
    \begin{itemize}
        \item vlab
        \item Vivado 2016.3
    \end{itemize}
    \section{实验核心实现}
    \begin{enumerate}
        \item 预测器主要的顶层设计逻辑是,在EX检查(检查部分的实现由一条语句在CPU最顶层实现)
        是否为Br指令,只有当是Br指令且预测失败时,由真正的Br信号确定PC,
        其余情况都由预测的信号确定(其余的情况包括不是Br指令和预测正确的情况)。
        \begin{lstlisting}[language=verilog]
always@(*)
begin
    if(need_update & (br_pred_out != is_br))
    begin
        branch_prediction_miss = 1;
        // if branch prediction fail, it'll depend on true br signal
        NPC = is_br ? PC_target : PC_out;
    end
    else
    begin
        branch_prediction_miss = 0;
        NPC = br_pred_in ? PC_pred : PC_query + 4;
    end
end
        \end{lstlisting}
        \item 实现一个记录预测结果和PC+4的buffer,长度为2,为了辅助在EX阶段的判断
        (存PC+4是为了在预测失败时且实际不跳转时使用)。
        \item 实现BTB(1bit预测器),主要实现状态机的转换(如下),
        是否命中、是否跳转及预测的PC等根据状态机的信息输出即可。
        \begin{lstlisting}[language=verilog]
integer i;
always@(posedge clk or posedge rst)
begin
    if(rst)
    begin
        for(i = 0; i < SET_SIZE; i = i + 1)
        begin
            TAG[i] <= 0;
            DATA[i] <= 0;
            VALID[i] <= 0;
            STATE[i] <= 0;
        end
    end
    else if(need_update)
    begin
        TAG[addr_update] <= tag_update;
        DATA[addr_update] <= update_data;
        VALID[addr_update] <= 1'b1;
        STATE[addr_update] <= is_br;
    end
end
        \end{lstlisting}
        \item 实现2bit BHT,主要实现状态机(参考课程ppt,00和01表示跳转,10和11表示不跳转,实现如下),
        是否跳转由所处状态决定。
        \begin{lstlisting}
integer i;
always@(posedge clk or posedge rst)
begin
    if(rst)
    begin
        for(i = 0; i < SET_SIZE; i = i + 1)
        begin
            STATE[i] <= 0;
        end
    end
    else
    begin
        if(need_update)
        begin
            if(is_br)
            begin
                STATE[addr_update] <= (STATE[addr_update] == 0 ? 0 : STATE[addr_update] - 1);
            end
            else
            begin
                STATE[addr_update] <= (STATE[addr_update] == MAX_VAL ? MAX_VAL : STATE[addr_update] + 1);
            end
        end
    end
end
        \end{lstlisting}
    \end{enumerate}
    \section{实验结果及分析}
    \subsection{分析分支收益和分支代价}
    \begin{itemize}
        \item 分支收益:一次分支预测命中可带来2个时钟周期的收益
        \item 分支代价:预测失败没有收益,代价为2个时钟周期(EX阶段检验,会影响IF和ID段已经执行的部分)
    \end{itemize}
    \subsection{统计未使用分支预测和使用分支预测的总周期数及差值}
    \begin{table}[H]
        \centering
        \begin{tabular}{ccccc}
            \hline
            ~ & btb.s & bht.s & QuickSort.s & MatMul.s \\
            \hline
            无分支预测 & 512 & 538 & 41859 & 297795 \\
            BTB分支预测(差值) & 316(196) & 382(156) & 41932(-73) & 295727(2068) \\
            BHT+BTB分支预测(差值) & 318(194) & 370(168) & 41156(703) & 294182(3613) \\
            \hline
        \end{tabular}
    \end{table}
    \subsection{统计分支指令数目、动态分支预测正确次数和错误次数}
    \begin{table}[H]
        \centering
        % \begin{tabular}{ccccc}
        %     \hline
        %     ~ & btb.s & bht.s & QuickSort.s & MatMul.s \\
        %     \hline
        %     分支指令数目 & 101 & 110 & 10198 & 4896 \\
        %     BTB分支预测正确次数/错误次数 & 99/2 & 88/22 & 8176/2022 & 4077/819 \\
        %     BHT+BTB分支预测正确次数/错误次数 & 98/3 & 95/15 & 8749/1449 & 4329/567 \\
        %     \hline
        % \end{tabular}
        \begin{tabular}{ccccc}
            \hline
            ~ & btb.s & bht.s & QuickSort.s & MatMul.s \\
            \hline
            分支指令数目 & 101 & 110 & 10548 & 4896 \\
            BTB分支预测正确次数/错误次数 & 99/2 & 88/22 & 8479/2069 & 4092/804 \\
            BHT+BTB分支预测正确次数/错误次数 & 98/3 & 95/15 & 8761/1787 & 4350/546 \\
            \hline
        \end{tabular}
    \end{table}
    \subsection{对比不同策略并分析以上几点的关系}
    \begin{enumerate}
        \item 使用动态分支预测通常会带来正收益。
        \item 但在某些情况下收益不高甚至会产生负收益。例如在QuickSort测试中,简单BTB预测反而会增大运行周期数。
        \item BHT分支预测效果通常好于BTB。
        \item 不同测试样例上不同分支结构带来的运行周期数优化效果差别显著,表明程序性能优化与程序本身(或者程序类型)有较强的相关性。
    \end{enumerate}
    \section{实验总结}
    \begin{enumerate}
        \item 在本次实验中,实现了BTB和BHT动态分支预测,对动态分支预测的原理也有了更深的理解。
        \item 通过各种程序测试和指标分析,体会到动态分支预测对程序性能的实际优化效果。
    \end{enumerate}
\end{document}
