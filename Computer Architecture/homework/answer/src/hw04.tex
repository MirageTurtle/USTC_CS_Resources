\documentclass{article}
\usepackage{ctex}
\usepackage{geometry}
\geometry{a4paper,scale=0.9}
\usepackage{enumitem}
\usepackage{amsmath}
\usepackage{amssymb}
\usepackage{graphicx}
\usepackage{float}
\usepackage{graphics}


\title{HW04}
\author{PB19071405\ 王昊元}
\date{2022 年 04 月 26 日}

\begin{document}
    \maketitle

    \begin{enumerate}[label=\arabic*.]
        \item \begin{enumerate}[label=\alph*.]
            \item 未进行调度的情况下和已调度的情况下的指令顺序及停顿如下:\\
            \begin{table}[H]
                \centering
                \begin{tabular}{ccc}
                    \hline
                    时钟周期数 & 未调度 & 已调度 \\
                    \hline
                    1 & DADDIU R4,R1,\#800 & DADDIU R4,R1,\#800 \\
                    2 & L.D F2,0(R1) & L.D F2,0(R1) \\
                    3 & Stall & L.D F6,0(R2) \\
                    4 & MUL.D F4,F2,F0 & MUL.D F4,F2,F0 \\
                    5 & L.D F6,0(R2) & DADDIU R1,R1,\#8 \\
                    6 & Stall & DADDIU R2,R2,\#8 \\
                    7 & Stall & DSLTU R3,R1,R4 \\
                    8 & Stall & Stall \\
                    9 & Stall & Stall \\
                    10 & ADD.D F6,F4,F6 & ADD.D F6,F4,F6 \\
                    11 & Stall & Stall \\
                    12 & Stall & Stall \\
                    13 & Stall & BNEZ R3,foo \\
                    14 & S.D F6,0(R2) & S.D F6,0(R2) \\
                    15 & DADDIU R1,R1,\#8 & ~ \\
                    16 & DADDIU R2,R2,\#8 & ~ \\
                    17 & DSLTU R3,R1,R4 & ~ \\
                    18 & Stall & ~ \\
                    19 & BNEZ R3,foo & ~ \\
                    20 & Stall & ~ \\
                    \hline
                \end{tabular}
            \end{table}
            未调度时,结果向量Y中每个元素的执行时间,也就是循环的时钟周期为19,调度后为13。\\
            为使处理器硬件独自匹配调度编译器所实现的性能改进,时钟频率应当为原来的$\frac{19}{13} = 1.46$倍。
            \item 展开3次可消除循环开销,指令调度结果如下:\\
            \begin{table}[H]
                \centering
                \begin{tabular}{cc}
                    \hline
                    时钟周期数 & 指令 \\
                    \hline
                    1 & DADDIU R4,R1,\#800 \\
                    2 & L.D F2,0(R1) \\
                    3 & L.D F6,0(R2) \\
                    4 & MUL.D F4,F2,F0 \\
                    5 & L.D F2,8(R1) \\
                    6 & L.D F10,8(R2) \\
                    7 & MUL.D F8,F2,F0 \\
                    8 & L.D F2,16(R1) \\
                    9 & L.D F14,16(R2) \\
                    10 & MUL.D F12,F2,F0 \\
                    11 & ADD.D F6,F4,F6 \\
                    12 & DADDIU R1,R1,\#24 \\
                    13 & ADD.D F10,F8,F10 \\
                    14 & ADD.D R2,R2,\#24 \\
                    15 & DSLTU R3,R1,R4 \\
                    16 & ADD.D F14,F12,F14 \\
                    17 & S.D F6,-24(R2) \\
                    18 & S.D F10,-16(R2) \\
                    19 & BNEZ R3,foo \\
                    20 & S.D F14,-8(R2) \\
                    \hline
                \end{tabular}
            \end{table}
            由上表可知,处理Y中的3个元素的执行时间为19个周期,即每个元素的执行时间为$\frac{19}{3}$个周期。
        \end{enumerate}
        \item 如下图所示:\\
        \begin{table}[H]
            \centering
            \begin{tabular}{cccccc}
                \hline
                迭代 & 指令 & 发射 & 执行/存储器访问 & 写CDB & 注释\\
                \hline
                1 & L.D F2,0(R1) & 1 & 2 & 3 & ~ \\
                1 & MUL.D F4,F2,F0 & 2 & 4 & 19(=4+15) & 等待F2写回 \\
                1 & L.D F6,0(R2) & 3 & 4 & 5 & ~ \\
                1 & ADD.D F6,F4,F6 & 4 & 20 & 30(=20+10) & 等待F4写回 \\
                1 & S.D F6,0(R2) & 5 & 31 & ~ & 等待F6写回 \\
                1 & DADDIU R1,R1,\#8 & 6 & 7 & 8(=7+1) & ~ \\
                1 & DADDIU R2,R2,\#8 & 7 & 8 & 9(=8+1) & ~ \\
                1 & DSLTU R3,R1,R4 & 8 & 9 & 10(=9+1) & ~ \\
                1 & BNEZ R3,foo & 9 & 11 & ~ & 等待R3写回 \\
                
                2 & L.D F2,0(R1) & 10 & 12 & 13 & 等待跳转结果 \\
                2 & MUL.D F4,F2,F0 & 11 & 19 & 34(=19+15) & 等待乘法器空闲 \\
                2 & L.D F6,0(R2) & 12 & 13 & 14 & ~ \\
                2 & ADD.D F6,F4,F6 & 13 & 35 & 45(=35+10) & 等待F4写回 \\
                2 & S.D F6,0(R2) & 14 & 46 & ~ & 等待F6写回 \\
                2 & DADDIU R1,R1,\#8 & 15 & 16 & 17(=16+1) & ~ \\
                2 & DADDIU R2,R2,\#8 & 16 & 17 & 18(=17+1) & ~ \\
                2 & DSLTU R3,R1,R4 & 17 & 18 & 19(=18+1) & ~ \\
                2 & BNEZ R3,foo & 18 & 20 & ~ & 等待R3写回 \\
                
                3 & L.D F2,0(R1) & 19 & 20 & 21 & ~ \\
                3 & MUL.D F4,F2,F0 & 20 & 34 & 49(=34+15) & 等待乘法器空闲 \\
                3 & L.D F6,0(R2) & 21 & 22 & 23 & ~ \\
                3 & ADD.D F6,F4,F6 & 22 & 50 & 60(=50+10) & 等待F4写回 \\
                3 & S.D F6,0(R2) & 23 & 61 & ~ & 等待F6写回 \\
                3 & DADDIU R1,R1,\#8 & 24 & 25 & 26(=25+1) & ~ \\
                3 & DADDIU R2,R2,\#8 & 25 & 26 & 27(=26+1) & ~ \\
                3 & DSLTU R3,R1,R4 & 26 & 27 & 28(=27+1) & ~ \\
                3 & BNEZ R3,foo & 27 & 29 & ~ & 等待R3写回 \\
                \hline
            \end{tabular}
        \end{table}
        PS: 因为EX和MEM在同一时钟周期完成,所以表中执行指令的周期和访问存储器的周期合并。\\
        \begin{itemize}
            \item 第1次迭代周期为31(=31-1+1)周期
            \item 第2次迭代周期为37(=46-10+1)周期
            \item 第3次迭代周期为43(=61-19+1)周期
        \end{itemize}
        \item 
        \begin{enumerate}[label=(\arabic*)]
            \item 
            \begin{align*}
                \text{CPI} &= \text{没有分支的CPI}+\text{分支预测的CPI} \\
                &= 1 \times 85\% + \text{平均一个分支分支预测的CPI} \times 15\% \\
                &= 0.85 + \text{命中的CPI + 未命中的CPI} \times 0.15 \\
                &= 0.85 + (90\% \times (90\% \times 1 + 10\% \times 4) + 10\% \times 3) \times 0.15 \\
                &= 1.0705
            \end{align*}
            \item 
            \begin{align*}
                \text{CPI'} &= \text{没有分支的CPI}+\text{分支预测的CPI} \\
                &= 1 \times 85\% + 2 \times 15\% \\
                &= 1.15
            \end{align*}
            采用分支目标缓冲执行速度更快。
        \end{enumerate}
        \item 因为只考虑无条件转移指令,所以当缓冲命中时,则等同于预测正确的情况。\\
        $$\text{CPI} = \text{没有分支的CPI}+\text{分支预测的CPI}$$
        由
        $$\text{CPI}_1 = 1 \times 95\% + \text{CPI}_\text{jump} \times 5\% = 1.1$$
        可知$\text{CPI}_\text{jump} = 3$\\
        则有
        $$\text{CPI}_2 = 1 \times 95\% + (90\% \times 1 + 10\% \times 3) \times 5\% = 1.01$$
    \end{enumerate}
\end{document}