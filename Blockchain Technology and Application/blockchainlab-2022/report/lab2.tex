\documentclass{article}
\usepackage{ctex}
\usepackage{amsmath}
\usepackage{amssymb}
\usepackage{graphicx}
\usepackage{float}
\usepackage{multirow}
\usepackage{verbatim}
\usepackage{subfigure}
\usepackage{color}
\usepackage{enumerate}
\usepackage{enumitem}
\usepackage{geometry}
\usepackage{listings}
\usepackage{fontspec}
\newfontfamily\jetbrains{JetBrains Mono}
\lstset{
    breaklines,                                 % 自动将长的代码行换行排版
    extendedchars=false,                        % 解决代码跨页时,章节标题,页眉等汉字不显示的问题
    backgroundcolor=\color[rgb]{0.96,0.96,0.96},% 背景颜色
    keywordstyle=\color{blue}\bfseries,         % 关键字颜色
    identifierstyle=\color{black},              % 普通标识符颜色
    commentstyle=\color[rgb]{0,0.6,0},          % 注释颜色
    stringstyle=\color[rgb]{0.58,0,0.82},       % 字符串颜色
    showstringspaces=false,                     % 不显示字符串内的空格
    numbers=left,                               % 显示行号
    numberstyle=\tiny\jetbrains,                    % 设置数字字体
    basicstyle=\small\jetbrains,                    % 设置基本字体
    captionpos=t,                               % title在上方(在bottom即为b)
    frame=single,                               % 设置代码框形式
    rulecolor=\color[rgb]{0.8,0.8,0.8},         % 设置代码框颜色
}
\usepackage{hyperref}
\hypersetup{
    hypertex=true,
    colorlinks=true,
    linkcolor=black,
    anchorcolor=black,
    citecolor=black,
}

\title{区块链实验二 实验报告}
\author{PB19071405 王昊元}
\date{2022 年 04 月 22 日}

\begin{document}
    \maketitle

    \section{实验目的及要求}
    \subsection{实验目的}
    \begin{itemize}
        \item 实现区块链上的POW证明算法
        \item 理解区块链上的难度调整的作用
    \end{itemize}
    \subsection{实验要求}
    \begin{itemize}
        \item 完成{\jetbrains proofofwork.go},包括{\jetbrains Run()}和{\jetbrains Validate()}两个函数
        \item 完成比较 {\jetbrains targetBits} 的修改带来的计算次数上的变化情况,从5-15的变化
    \end{itemize}
    \section{实验原理}
    \subsection{POW}
    工作量证明(Proof-of-Work,PoW)是一种对应服务与资源滥用、或是拒绝服务攻击的经济对策。

    一般要求用户进行一些耗时适当的复杂运算,并且答案能被服务方快速验算,以此耗用的时间、设备与能源做为担保成本,以确保服务与资源是被真正的需求所使用。
    \subsection{区块链哈希} \label{blockchain hash}
    比特币采用了哈希现金(hashcash)的工作量证明机制,对应流程如下:
    \begin{enumerate}[label=\arabic*.]
        \item 首先构建当前区块头,区块头包含上⼀个区块哈希值(32位),当前区块数据对应哈希(32位,即区块数据的merkle根),时间戳,区块难度,计数器(nonce)。
        \item 添加计数器,作为随机数。计算器从0开始基础,每个回合加1。
        \item 对于上述的数据来进行一个哈希的操作。
        \item 判断结果是否满足计算的条件:如果符合,则得到了满足结果;如果没有符合,从2开始重新直接2、3、4步骤。
    \end{enumerate}
    \section{实验平台}
    \begin{itemize}
        \item 操作系统:macOS Big Sur 11.2.3
        \item Go version: go1.17.8 darwin/amd64
    \end{itemize}
    \section{实验步骤}
    \subsection{\jetbrains proofofwork.go}
    \begin{itemize}
        \item {\jetbrains Run()}
        按照第\ref{blockchain hash}部分的算法,在 {\jetbrains nonce < maxNonce}时不断循环计算哈希,直到满足要求。
        代码实现如下:
        \begin{lstlisting}[language=go]
func (pow *ProofOfWork) Run() (int, []byte) {
	nonce := 0
	var hash_int big.Int
	var hash [32]byte

	for nonce < maxNonce {
		// data waiting for hash
		data := bytes.Join(
			[][]byte{
				pow.block.PrevBlockHash,
				pow.block.HashData(),
				IntToHex(pow.block.Timestamp),
				IntToHex(int64(pow.block.Bits)),
				IntToHex(int64(nonce)),
			},
			[]byte{},
		)
		// compute hash and the corrsponding big int
		hash = sha256.Sum256(data)
		hash_int.SetBytes(hash[:])
		// compute the target int (a number begin with pow.block.Bits zero in binary)
		target := big.NewInt(1)
		target.Lsh(target, uint(256 - pow.block.Bits))
		// validate
		if hash_int.Cmp(target) == -1 {
			break
		} else {
			nonce += 1
		}
	}
	// return nonce, pow.block.Hash
	return nonce, hash[:]
}
        \end{lstlisting}
        \item {\jetbrains Validate()}
        直接判断区块头的哈希是否满足要求。
        \begin{lstlisting}[language=go]
func (pow *ProofOfWork) Validate() bool {
	var hash_int big.Int
	// compute the big int corresponding block hash
	hash_int.SetBytes(pow.block.Hash[:])
	// compute the target int
	target := big.NewInt(1)
	target.Lsh(target, uint(256 - pow.block.Bits))	
	// validate
	return (hash_int.Cmp(target) == -1)
}
        \end{lstlisting}
    \end{itemize}
    \section{实验结果}
    向区块链中插入数据,结果如下:
    \begin{figure}
        \centering
        \includegraphics*[width=0.7\textwidth]{./figs/pow_result1.jpg}
        \includegraphics*[width=0.7\textwidth]{./figs/pow_result2.jpg}
        \caption{实验结果截图}
        \label{result screenshot}
    \end{figure}

    从图\ref{result screenshot}中可以看到,数据成功的被加密,{\jetbrains TargetBits}随着链长的增加逐渐增加,
    整体上{\jetbrains Nonce}的大小也随着{\jetbrains TargetBits}的增加而增加,
    {\jetbrains Pow}部分变为了{\jetbrains True}。

    {\jetbrains Nonce}随{\jetbrains TargetBits}的变化如下图所示:
    \begin{figure}
        \centering
        \includegraphics*[width=0.7\textwidth]{./figs/targetbits-nonce.jpg}
        \caption{TargetBits - Nonce 图}
        \label{targetbits-nonce}
    \end{figure}
    从图\ref{targetbits-nonce}中能看出由于{\jetbrains TargetBits}比较小时,符合要求的数较多,
    导致{\jetbrains Nonce}会有波动,不过从整体上{\jetbrains Nonce}的大小还是随着{\jetbrains TargetBits}的增大而增大。
\end{document}
